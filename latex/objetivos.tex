\chapter{Objetivos d trabalho}
1. Introdução sobre o sistema que será simulado\\
2. Diagrama de bloco da pisco de luz\\
3. MPPT\\
4. Simulação de Bateria - Estagios de carga de 18650\\
5. Buck com GAN\\
6. interessante não simular controle e sim apenas os 3 pontos importantes de carga: Fonte de corrente, Fonte de tensão\\
7. Usar indutor acoplado para amostrar a corrente\\
0. colocar primeiro: Estudar a relação de beneficios em se utilizar dispositivos GaN ao invés de Silício. Ainda avaliar, se possível, seu custo e viabilidade atuais, no contexto de fontes chaveadas. Especificamente podemos escolher uma topologia simples para base de comparação.

\begin{center}
\begin{circuitikz}
%Coluna,Linha
%(0,0) botton left
\draw %painel solar
    [dashed] (-1,-1)   to [short,i=] (6,-1)
             (6,-1)   to [short] (6,5)
             (6,5)   to [short] (-1,5)
             (-1,5)   to [short] (-1,-1)
            (3,-0.8) node[]{Painel Solar Fotovoltaico}
;

\draw %buck
    [dashed] (7,-1)   to [short,i=] (12,-1)
             (12,-1)  to [short] (12,5)
             (12,5)   to [short] (7,5)
             (7,5)   to [short] (7,-1)
            (9.5,-0.8) node[]{Conversor Buck}
;

\draw %bat
    [dashed] (13,-1)   to [short,i=] (15,-1)
             (15,-1)  to [short] (15,5)
             (15,5)   to [short] (13,5)
             (13,5)   to [short] (13,-1)
             (14,-0.8) node[]{Bateria}
;


        

    
\draw
    (0,0)   to [short,-*] (6,0)
    (3.5,4) to [R, l_=$R_s$] (5.5,4)
    (5.5,4) to [short,-*] (6,4)
    (0,0)   to [american current source, l_=$I_{pv}$] (0,4)
    (0,4)   to [short] (3.5,4)
    (1.5,4) to [short, *-, i=$I_d$] (1.5,3) 
    (1.5,3) to [D] (1.5,1)
    (1.5,1) to [short,-*] (1.5,0) 
    (3,4)   to [R, *-*, l_=$R_p$] (3,0)
    ; 
\draw
    %(6,0)   to [short,*-] (7,4)
    %(7,4)   to [short,*-] (8,4)
    (6,4)   to [short,*-] (8,4)
    (8,4)   to [switch, l_=$GAN$] (10,4)
    %(6,4)   to [short,*-] (7,0)
    %(7,0)   to [short,*-] (14,0)
    (6,0)   to [short,*-] (12,0)
    (12,0)   to [short,*-*] (14,0) 
    (8,0)   to [C,*-*] (8,4)
    (10,0)  to [D,*-*] (10,4)
    (10,4)  to [L,l_=$L$] (12,4) 
    (12,4)   to [short,*-] (14,4) 
    (14,4)  to [battery,-] (14,0)
    node[ground]{}
;
    
\end{circuitikz}
\end{center}



\section{Fonte chaveada de referência}
O objetivo é comparar as propriedades dos dispositivos e como eles afetam uma dada fonte chaveada de referência. O controle da fonte deverá permanecer o mesmo durante a comparação ou até mesmo não ser utilizado. Pode-se analisar a fonte apenas com carga resistiva e em regime permanente. Como exemplo, utilizei um circuito elevador de tensão (boost) bastante simples e sem controle. O circuito está na Figura \ref{FigCircuit} e foi simulado no LTSpice, da Analog Devices \cite{ltspice}.
\begin{figure}[H]
\caption{Circuito de referência: Boost}
 \centering % para centralizarmos a figura
\includegraphics[width=14cm]{figuras/1.JPG} 
\label{FigCircuit}
\end{figure}
\noindent O circuito superior da Figura \ref{FigCircuit} utiliza uma chave simples no lugar de um transistor, com $R_{DS_{(on)}}$ de um transistor GaN comercial. O circuito da parte de baixo da mesma figura, utiliza um $R_{DS_{(on)}}$ de um transistor Mosfet baseado em Silício.\par
\noindent Na figura \ref{FigCircuitVoutIout}, temos a saída dos dois circuitos, $V_{(out)}$ e $I_{(out)}$. Nota-se um grande \textit{overshoot}, dado que não há controle.
\begin{figure}[H]
\caption{Circuito de referência: Tensões e correntes de saída}
 \centering % para centralizarmos a figura
\includegraphics[width=14cm]{figuras/2.JPG} 
\label{FigCircuitVoutIout}
\end{figure}
\noindent Pode-se, então, comparar os rendimentos dos circuitos, baseados apenas nessa característica, $R_{DS_{(on)}}$. Na Figura \ref{FigCircuitEff}, há o calculo feito pelo software. 
\begin{figure}[H]
\caption{Circuito de referência: Comparação da eficiência}
 \centering % para centralizarmos a figura
\includegraphics[width=8cm]{figuras/3.JPG} 
\label{FigCircuitEff}
\end{figure}
\noindent A eficiência desse \textit{Boost} utilizado GaN é de $91,607\%$, enquanto com Silício é de $90,67\%$. Embora pareça pouco a diferença de $~1\%$, deve-se notar que este é um circuito elevador de $5V$ para $10V$, com corrente $I_o$ médio de apenas $0,5A$. Para escalas maiores de potência, $kW$ ou até $MW$, as perdas serão muito maiores.

\section{Oportunidades de publicação}
As pesquisas feitas através do site da IEEE (Journals: \textit{IEEE Transactions on Power Electronics} e \textit{IEEE Journal of Emerging and Selected Topics in Power Electronics}) mostraram que GaN é um tópico que vem crescendo ao longo dos anos, porém ainda com pouco número de publicações. No grafico da Figura \ref{FigPublications}, há o número de publicações, por ano, no IEEE, nos \textit{journals} citados acima. Na revista da Sobraep só foram encontrados dois artigos sobre o tema. 
\begin{figure}[H]
\caption{Número de publicações com os termos "GaN Hemt", por ano}
 \centering % para centralizarmos a figura
\includegraphics[width=14cm]{figuras/IEEE_Graf.JPG}
\label{FigPublications}
\end{figure}
\noindent O número crescente de publicações, indica que o interesse nesse tipo de dispositivo semicondutor cresce na mesma medida. Alguns artigos servirão de norte para este projeto: \cite{Chu}, \cite{Huang}, \cite{mitova}.



