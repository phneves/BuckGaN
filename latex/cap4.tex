\chapter{Plano de trabalho e cronograma}
\label{chapterCronograma}
\par Paragrafo explicando o cronograma.
%adicionar algum comentario sobre a linha do tempo
\begin{figure}[H]
\caption{Cronograma} 
\begin{center}
\begin{circuitikz}
%draw horizontal line
\draw (0,0) -- (14,0);
\draw [dashed](14,0) -- (15,0);
%draw vertical lines
\foreach \x in {0,1,3,5,7,10.5,14}
\draw (\x cm,3pt) -- (\x cm,-3pt);

%draw nodes
\draw (0,0) node[below=3pt] {$ 2020 $} node[above=3pt] {$   $};
\draw (1,0) node[below=3pt] {$  $} node[above=3pt] {$ I $};
\draw (2,0) node[below=3pt] {$  $} node[above=3pt] {$  $};
\draw (3,0) node[below=3pt] {$  $} node[above=3pt] {$ II $};
\draw (4,0) node[below=3pt] {$  $} node[above=3pt] {$  $};
\draw (5,0) node[below=3pt] {$  $} node[above=3pt] {$  $};
\draw (6,0) node[below=3pt] {$  $} node[above=3pt] {$  $};
\draw (7,0) node[below=3pt] {$ 2021 $} node[above=3pt] {$ III $};
\draw (10.5,0) node[below=3pt] {$ 07/2021 $} node[above=3pt] {$ IV $};
\draw (14,0) node[below=3pt] {$ 2022 $} node[above=3pt] {$ V $};
\end{circuitikz}
\end{center}
\label{figCronograma}
\end{figure}

\par O cronograma da figura \ref{figCronograma} está detalhado a seguir:
\renewcommand{\labelenumi}{\Roman{enumi}}
 \begin{enumerate}
   \item Início do programa
   \item Disciplinas:
    \begin{itemize}
        \item IT505 - Fontes chaveadas
        \item IE320 - Tópicos em Eletrônica I: Introdução a compatibilidade eletromagnética
        \item IT302 - Eletronica de potência I
        \item IT306 - Tópicos em Sistemas de Energia Elétrica III
    \end{itemize}
   \item Inicio das simulações e validações de modelos
   \item Qualificação
   \item Integração dos modelos simulados e apresentação da dissertação

 \end{enumerate}
 