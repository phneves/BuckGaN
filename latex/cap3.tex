\chapter{Metodologia utilizada}
\label{chapterMetodologia}
%1. Introdução sobre o sistema que será simulado\\
%2. Diagrama de bloco da pisco de luz\\
%3. MPPT\\
%4. Simulação de Bateria - Estagios de carga de 18650\\
%5. Buck com GAN\\
%%6. interessante não simular controle e sim apenas os 3 pontos importantes de carga: Fonte de corrente, Fonte de tensão\\%
%7. Usar indutor acoplado para amostrar a corrente\\
%0. colocar primeiro: Estudar a relação de beneficios em se utilizar dispositivos GaN ao invés de Silício. Ainda avaliar, se possível, seu custo e viabilidade atuais, no contexto de fontes chaveadas. Especificamente podemos escolher uma topologia simples para base de comparação.
%encontrar nome da descontinuidade no gate do GaN
\par Neste trabalho, toda a análise será fundamentada em simulação computacional utilizando o software LTSpice, da Analog Devices \cite{ltspice}. Os circuitos analisados estão representados na figura \ref{figPVBuckBat} e serão analisados separadamente. \footnote{Os circuitos simulados aqui se encontram disponíveis para download no repositório do autor: \url{www.github.com/phneves/BuckGaN}}

\begin{figure}[H]
\caption{Modelo de simulação completo} 
\begin{center}
\begin{circuitikz}
%Coluna,Linha
%(0,0) botton left
\draw %painel solar
    [dashed] (-1,-1)   to [short,i=] (6,-1)
             (6,-1)   to [short] (6,5)
             (6,5)   to [short] (-1,5)
             (-1,5)   to [short] (-1,-1)
            (3,-0.8) node[]{Painel Solar Fotovoltaico};
\draw %buck
    [dashed] (7,-1)   to [short,i=] (12,-1)
             (12,-1)  to [short] (12,5)
             (12,5)   to [short] (7,5)
             (7,5)   to [short] (7,-1)
            (9.5,-0.8) node[]{Conversor Buck};
\draw %bat
    [dashed] (13,-1)   to [short,i=] (15,-1)
             (15,-1)  to [short] (15,5)
             (15,5)   to [short] (13,5)
             (13,5)   to [short] (13,-1)
             (14,-0.8) node[]{Bateria};
\draw
    (0,0)   to [short,-*] (6,0)
    (3.5,4) to [R, l_=$R_s$] (5.5,4)
    (5.5,4) to [short,-*] (6,4)
    (0,0)   to [american current source, l_=$I_{pv}$] (0,4)
    (0,4)   to [short] (3.5,4)
    (1.5,4) to [short, *-, i=$I_d$] (1.5,3) 
    (1.5,3) to [D] (1.5,1)
    (1.5,1) to [short,-*] (1.5,0) 
    (3,4)   to [R, *-*, l_=$R_p$] (3,0); 
\draw
    %(6,0)   to [short,*-] (7,4)
    %(7,4)   to [short,*-] (8,4)
    (6,4)   to [short,*-] (8,4)
    %(8,4)   to [switch, l_=$GaN$] (10,4)
    (10,4) to[Tnjfet,n=jfet] (8,4)
    %(6,4)   to [short,*-] (7,0)
    %(7,0)   to [short,*-] (14,0)
    (6,0)   to [short,*-] (12,0)
    (12,0)   to [short,*-*] (14,0) 
    (8,0)   to [C,*-*] (8,4)
    (10,0)  to [D,*-*] (10,4)
    (10,4)  to [L,l_=$L$] (12,4) 
    (12,4)   to [short,*-] (14,4) 
    (14,4)  to [battery,-] (14,0)
    node[ground]{};
\end{circuitikz}
\end{center}
\label{figPVBuckBat}
\end{figure}

\section{Conversor Buck}
Considere as topologias apresentadas na na figura \ref{figBuckDiodoGan}. O conversor Buck será analisado em 4 configurações: 1. GaN(EPC2001) + Diodo; 2. 2xGaN; 3. Mosfet(IRF530) + Diodo; 4. 2xMosfet(IRF530).
\begin{figure}[H]
\caption{Model de simulação completo} 
\begin{center}
\begin{circuitikz}
%Coluna,Linha
%(0,0) botton left

\draw
    (0,4)   to [american voltage source, l_=$V_{in}$] (0,0)
    (0.6,4.2) node[]{$V_{in}$}
    (0,4)   to [short,-] (1,4)
    (3,4) to[Tnjfet,n=jfet] (1,4)
    %(6,4)   to [short,*-] (7,0)
    %(7,0)   to [short,*-] (14,0)
    (0,0)   to [short,-] (7,0)
    %(12,0)   to [short,-] (15,0) 
    (1.2,0)   to [C,l_=$C_{in}$, *-*] (1.2,4)
    %(3,0)  to [D,l_=$D$,*-*,color=red] (3,4)
    (3,4)  to [L,l_=$L$,i=$I_o$] (5,4) 
    (5,4)   to [short,-] (7,4) 
    (5.5,4)   to [C,l_=$C_{out}$,*-*] (5.5,0)
    (6,4.2) node[]{$V_{o}$}
    (7,4)  to [R,l_=$R_{o}$,-*] (7,0)
    node[ground]{};
    
    \draw
    (8,4)   to [american voltage source] (8,0)
    (8.6,4.2) node[]{$V_{in}$}
    (8,4)   to [short,-] (9,4)
    (11,4) to[Tnjfet,n=jfet] (9,4)
    %(6,4)   to [short,*-] (7,0)
    %(7,0)   to [short,*-] (14,0)
    (8,0)   to [short,-] (15,0)
    %(12,0)   to [short,-] (15,0) 
    (9.2,0)   to [C, *-*] (9.2,4)
    %(11,0)  to [D,l_=$D$,*-*] (11,4)
    %(11,0) to[Tnjfet,n=jfet,color=red, *-*] (11,4)
    (11,4)  to [L,l_=$L$,i=$I_o$] (13,4) 
    (13,4)   to [short,-] (15,4) 
    (13.5,4)   to [C,l_=$C_{out}$,*-*] (13.5,0)
    (14,4.2) node[]{$V_{o}$}
    (15,4)  to [R,l_=$R_{o}$,-*] (15,0)
    node[ground]{};
    
\draw 
[red] (11,0) to[Tnjfet,n=jfet,color=red, *-*] (11,4)
(3,0)  to [D,l_=$D$,*-*,color=red] (3,4)
;
\end{circuitikz}
\end{center}
\label{figBuckDiodoGan}
\end{figure}
\par Os componentes EPC2001 e IRF530 estão detalhados na tabela \ref{t_ComparaTransistor} \cite{IRF530},\cite{epc2001C}.
\begin{table}[H]
\centering
\caption{Transistores usados nas simulações}
\begin{tabular}{llllll}
\hline
Transistor & $I_d$                     & $V_{gs}$ & $V_{ds}$ & $Rds_{on}$ & $Q_G$  \\ \hline
ECP2001C   & $36A$                     & $5V$  & $100V$                        & $5,6m\Omega $                          & $7.5nC$             \\
IRF530     &  $14A$                      & $5V$  & $100V$                        & $160m\Omega $                          & $26nC$              \\
\hline
\end{tabular}
\label{t_ComparaTransistor}
\end{table}

\par Variando-se a frequência de trabalho, foi calculado os valores dos componentes e \textit{duty-cycle}($\delta$) para cada uma das topologias, seguindo as equações descritas na seção \ref{sectionCap2Buck}. Os valores se encontram na tabela \ref{t_VariacaoFrequencia}.

\begin{table}[H]
\centering
\caption{Valores para conversor Buck variando-se a frequência de trabalho}
\begin{tabular}{llllllllll}
\hline
$f$  &	$V_o$	& $V_{in}$ &	$\delta$	& L & Periodo ($\tau$)	& $I_o$	& $\Delta V_o (mV)$	& $C_o$ & 	Ro($\Omega$)  \\ \hline
$ 25kHz$	& $4,3V$	& $17V$	& $25\%$	& $107\mu H$	& $40 \mu s$& $0,6A$	& $600mV$	& $10\mu F$	& $7,15 \Omega$  \\
$ 50kHz$    & $4,3V$    & $17V$	& $25\%$	& $53,5\mu H$	& $20\mu s$	& $0,6A$	& $300mV$	& $10\mu F$	& $7,15 \Omega$  \\
$100kHz$	& $4,3V$	& $17V$	& $25\%$	& $26,8\mu H$	& $10\mu s$	& $0,6A$	& $150mV$	& $10\mu F$	& $7,15 \Omega$  \\
$200kHz$	& $4,3V$	& $17V$	& $25\%$	& $13,4\mu H$	& $5\mu s$	& $0,6A$	& $75mV$	& $10\mu F$	& $7,15 \Omega$  \\
$300kHz$	& $4,3V$	& $17V$	& $25\%$	& $8,92\mu H$	& $3,33\mu s$& $0,6A$	& $50mV$	& $10\mu F$	& $7,15 \Omega$  \\
$400kHz$	& $4,3V$	& $17V$	& $25\%$	& $6,69\mu H$	& $2,5\mu s$ & $0,6A$	& $37mV$	& $10\mu F$	& $7,15 \Omega$  \\
$500kHz$	& $4,3V$	& $17V$	& $25\%$	& $5,35\mu H$	& $2\mu s$	 & $0,6A$	& $30mV$	& $10\mu F$	& $7,15 \Omega$  \\ \hline
\end{tabular}
\label{t_VariacaoFrequencia}
\end{table}

\par Variando-se a tensão de entrada $V_{in}$, com frequência $f=100kHz$, foi calculado os valores dos componentes e \textit{duty-cycle}($\delta$) para cada uma das topologias, seguindo as equações descritas na seção \ref{sectionCap2Buck}. Os valores se encontram na tabela \ref{t_VariacaoTensao}.

\begin{table}[H]
\centering
\caption{Valores para conversor Buck variando-se a tensão de entrada}
\begin{tabular}{lllllllll}
\hline
$V_{in}$ &	$V_o$	& 	$\delta$ &	L &  $I_o$	& $C_o$ & 	Ro($\Omega$)  \\ \hline
$13V$	& $4,3V$ & $33,07\% $ &  $24\mu H$	& 0,6	& $10 \mu F$	& $7,15 \Omega$ \\
$15V$	& $4,3V$ & $28,66\% $ &  $25,6\mu H$	& 0,6	& $10 \mu F$	& $7,15 \Omega$ \\
$17V$	& $4,3V$ & $25,29\% $ &  $26,8\mu H$	& 0,6	& $10 \mu F$	& $7,15 \Omega$ \\ 
\hline
\end{tabular}
\label{t_VariacaoTensao}
\end{table}

\section{Painel solar fotovoltaico}
\par O Painel solar usado nas simulações foi baseado no circuito da figura \ref{figPVsim} e os resistores $R_s$ e $R_p$ calculados como em \ref{Rp}. Os valores estão presentes na tabela \ref{t_PV}

\begin{table}[H]
\centering
\caption{Valores para o modelo de painel solar fotovoltaico KM(P)10 de 10W}
\begin{tabular}{ll}
\hline
Parâmetro & Valor \\\hline
$V_{mp}$       & $17,56V$                    \\
$I_{sc}$       & $0,66A$                      \\
$I_{mp}$       & $0,6A$                       \\
$V_{oc}$       & $21,52V$                     \\
$R_p$       & $286\Omega$               \\
$R_s$        & $0,01\Omega$                    \\
\hline
\end{tabular}
\label{t_PV}
\end{table}

